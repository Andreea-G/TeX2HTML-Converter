% ------------------------------ html_commands.tex ------------------------------

% This file is called by the main tex file (say main_file.tex) file via % ------------------------------ html_commands.tex ------------------------------

% This file is called by the main tex file (say main_file.tex) file via % ------------------------------ html_commands.tex ------------------------------

% This file is called by the main tex file (say main_file.tex) file via % ------------------------------ html_commands.tex ------------------------------

% This file is called by the main tex file (say main_file.tex) file via \include{html_commands.tex}
% The file has dual purpose:
% - when placed in the same folder as the tex2html program, it contains only "Html specific commands", and no "Pdf specific commands" . Tex2html will make use of the "Html commands" and generate an html file.
% - when a pdf file is generated, the program automatically concatenates this file with pdf_commands.tex, which contains "Pdf specific commands". The resulting combined file *must* also be called html_commands.tex (so that it would be recognized by main_file.tex). The program then places the resulting file html_commands.tex obtained after concatenation in the same folder as main_file.tex. Then it generates a pdf file by simply running pdflatex in the usual manner (Pdf commands override some of the Html commands to generate a good pdf). Once can then also open the main_file.tex in a tex editor and generate a pdf as usual.
% Html commands will therefore be present in either case, but Pdf commands will only be present in the file located in the same folder as main_file.tex


% Short-hand for labeling things using hyperref (first is for numbering equations, second is for the rest: figures, sections etc).
\newcommand\hyeqref[1]{\hyperref[#1]{\eqref{#1}}}
\newcommand\hyref[1]{\hyperref[#1]{\ref{#1}}}

% ------- Html specific commands: -------
\renewcommand\cite[1]{[#1]}
\newcommand\citet[1]{[#1]}

% Redefine some common latex commands to work with tex2html. 
\renewcommand\indent{IndentHere }
\let\oldbs=\boldsymbol
\renewcommand\boldsymbol[1]{\mathit{\mathbf{#1}}}
%\newcommand\bs[1]{\boldsymbol{#1}}
\let\bs=\boldsymbol
\newcommand\DrawHLine{DrawHLine}
%\renewcommand\[{\begin{equation*}}
%\renewcommand\]{\end{equation*}}

% Usage: just like \caption but for videos. Writes Video: rather than Figure: in the beginning. 
\newcommand\videocaption[1]%
{\renewcommand{\figurename}{Video}%
\caption{#1}
}

% For including videos (similarly to figures).
% Usage: \includevideo{name} . Use same as you would use \includegraphics but do *NOT* write an extension. File must be name.ogg to be viewed with Firefox and name.mp4 to be viewed with Safari. Example:
%\begin{figure}[h!] \label{mov_bear}
%\centering
%\includevideo{bear_movie}
%\caption{Some movie of a bear.}
%\end{figure}
\newcommand\includevideo[1]%
{BeginVideo #1.ogg%
\includegraphics{fig_1.png}%
EndVideo
}

% Toggle environment
% Usage: \begin{htmlToggle}[What to show/hide]{Show/Hide}{label}[background color] Text to be shown \end{htmlToggle}
% Color codes: 6 hexadecimal numbers representing RGB. Eg. red = FF0000, yellow = FFFF00, black = 000000 etc  (see http://html-color-codes.info/)
% Here you can define your own color codes as below
\usepackage{xargs}
\def \White{FFFFFF}
\def \ZerothGray{FDFDFD}
\def \FirstGray{FAFAFA}
\def \SecondGray{F8F8F8}
\def \ThirdGray{F2F2F2}
\def \MyRed{FF0000}
\newenvironmentx{htmlToggle}[4][1=null, 4=\ZerothGray]% ----> SET DEFAULT COLOR HERE!!
{BeginToggle #2 #3 #4 #1 BeginToggle}%
{EndToggle}

% Alert environment
% Usage: \begin{htmlAlert}{Alert contents} Stuff to click on \end{htmlAlert}
% Examples may include math, links, text of several lines etc:
%\begin{htmlAlert}{\href{myhtml.html}{Url} } Click here! \end{htmlAlert} \\
%\begin{htmlAlert}{\href{http://www.google.com}{Google!} } Click here! \end{htmlAlert} \\
%\begin{htmlAlert}{Some text\\ some more boldsymbol $\boldsymbol{x}\boldsymbol{y}$  \\ and more and more and more and more and more and more and more and more and more and more and more and more and more } Click me! \end{htmlAlert} \\
%\begin{htmlAlert}{$x^\mu\arr x\prime^\mu$} Click me too! \end{htmlAlert} \\
\newenvironment{htmlAlert}[1]% 
{ BeginAlert #1 AlertOnClick}%
{EndAlert}

% Surprisingly this had not been defined. So we're defining it here.  
\newenvironment{fulljustify}%
{FullJustify}%
{EndJustify}

% Create multiple columns in the html file.
% Usage: \begin{htmlColumns}{# of columns} Text to be placed in columns \end{htmlColumns}
\newenvironment{htmlColumns}[1]%
{BeginColumns #1}%
{EndColumns  \DrawHLine}

%TODO: These are not added to the code yet. Will follow soon.
% % Wick contraction using "simplewick" package. Redefine simplewick's command "contraction" to work for html.
% \usepackage{simplewick}
% \let\oldContraction\contraction
% \renewcommand{\contraction}[5][5px]{%
%  \text{BeginWick}  #2 \text{StartWick}  #3 \text{MiddleWick}  #4 \text{FinishWick}  #5 \text{SizeWick} \text{#1} \text{EndWick}%
% }
% \newenvironment{simplewick}%
% {\text{BeginSimpleWick}}%
% {\text{EndSimpleWick}}
% 
% % Wick contraction using Dan Schroeder Macros.
% \newcommand{\beC}[2]{%
% \text{BeginSchroederWick} \text{StartWick} #2 \text{SizeWick} #1 \text{EndSchroederWick}%
% }
% \newcommand{\coC}[1]{%
% \text{BeginSchroederWick} \text{MiddleWick} #1 \text{EndSchroederWick}%
% }
% \newcommand{\enC}[2]{%
% \text{BeginSchroederWick} \text{FinishWick} #2 \text{SizeWick} #1 \text{EndSchroederWick}%
% }


% The file has dual purpose:
% - when placed in the same folder as the tex2html program, it contains only "Html specific commands", and no "Pdf specific commands" . Tex2html will make use of the "Html commands" and generate an html file.
% - when a pdf file is generated, the program automatically concatenates this file with pdf_commands.tex, which contains "Pdf specific commands". The resulting combined file *must* also be called html_commands.tex (so that it would be recognized by main_file.tex). The program then places the resulting file html_commands.tex obtained after concatenation in the same folder as main_file.tex. Then it generates a pdf file by simply running pdflatex in the usual manner (Pdf commands override some of the Html commands to generate a good pdf). Once can then also open the main_file.tex in a tex editor and generate a pdf as usual.
% Html commands will therefore be present in either case, but Pdf commands will only be present in the file located in the same folder as main_file.tex


% Short-hand for labeling things using hyperref (first is for numbering equations, second is for the rest: figures, sections etc).
\newcommand\hyeqref[1]{\hyperref[#1]{\eqref{#1}}}
\newcommand\hyref[1]{\hyperref[#1]{\ref{#1}}}

% ------- Html specific commands: -------
\renewcommand\cite[1]{[#1]}
\newcommand\citet[1]{[#1]}

% Redefine some common latex commands to work with tex2html. 
\renewcommand\indent{IndentHere }
\let\oldbs=\boldsymbol
\renewcommand\boldsymbol[1]{\mathit{\mathbf{#1}}}
%\newcommand\bs[1]{\boldsymbol{#1}}
\let\bs=\boldsymbol
\newcommand\DrawHLine{DrawHLine}
%\renewcommand\[{\begin{equation*}}
%\renewcommand\]{\end{equation*}}

% Usage: just like \caption but for videos. Writes Video: rather than Figure: in the beginning. 
\newcommand\videocaption[1]%
{\renewcommand{\figurename}{Video}%
\caption{#1}
}

% For including videos (similarly to figures).
% Usage: \includevideo{name} . Use same as you would use \includegraphics but do *NOT* write an extension. File must be name.ogg to be viewed with Firefox and name.mp4 to be viewed with Safari. Example:
%\begin{figure}[h!] \label{mov_bear}
%\centering
%\includevideo{bear_movie}
%\caption{Some movie of a bear.}
%\end{figure}
\newcommand\includevideo[1]%
{BeginVideo #1.ogg%
\includegraphics{fig_1.png}%
EndVideo
}

% Toggle environment
% Usage: \begin{htmlToggle}[What to show/hide]{Show/Hide}{label}[background color] Text to be shown \end{htmlToggle}
% Color codes: 6 hexadecimal numbers representing RGB. Eg. red = FF0000, yellow = FFFF00, black = 000000 etc  (see http://html-color-codes.info/)
% Here you can define your own color codes as below
\usepackage{xargs}
\def \White{FFFFFF}
\def \ZerothGray{FDFDFD}
\def \FirstGray{FAFAFA}
\def \SecondGray{F8F8F8}
\def \ThirdGray{F2F2F2}
\def \MyRed{FF0000}
\newenvironmentx{htmlToggle}[4][1=null, 4=\ZerothGray]% ----> SET DEFAULT COLOR HERE!!
{BeginToggle #2 #3 #4 #1 BeginToggle}%
{EndToggle}

% Alert environment
% Usage: \begin{htmlAlert}{Alert contents} Stuff to click on \end{htmlAlert}
% Examples may include math, links, text of several lines etc:
%\begin{htmlAlert}{\href{myhtml.html}{Url} } Click here! \end{htmlAlert} \\
%\begin{htmlAlert}{\href{http://www.google.com}{Google!} } Click here! \end{htmlAlert} \\
%\begin{htmlAlert}{Some text\\ some more boldsymbol $\boldsymbol{x}\boldsymbol{y}$  \\ and more and more and more and more and more and more and more and more and more and more and more and more and more } Click me! \end{htmlAlert} \\
%\begin{htmlAlert}{$x^\mu\arr x\prime^\mu$} Click me too! \end{htmlAlert} \\
\newenvironment{htmlAlert}[1]% 
{ BeginAlert #1 AlertOnClick}%
{EndAlert}

% Surprisingly this had not been defined. So we're defining it here.  
\newenvironment{fulljustify}%
{FullJustify}%
{EndJustify}

% Create multiple columns in the html file.
% Usage: \begin{htmlColumns}{# of columns} Text to be placed in columns \end{htmlColumns}
\newenvironment{htmlColumns}[1]%
{BeginColumns #1}%
{EndColumns  \DrawHLine}

%TODO: These are not added to the code yet. Will follow soon.
% % Wick contraction using "simplewick" package. Redefine simplewick's command "contraction" to work for html.
% \usepackage{simplewick}
% \let\oldContraction\contraction
% \renewcommand{\contraction}[5][5px]{%
%  \text{BeginWick}  #2 \text{StartWick}  #3 \text{MiddleWick}  #4 \text{FinishWick}  #5 \text{SizeWick} \text{#1} \text{EndWick}%
% }
% \newenvironment{simplewick}%
% {\text{BeginSimpleWick}}%
% {\text{EndSimpleWick}}
% 
% % Wick contraction using Dan Schroeder Macros.
% \newcommand{\beC}[2]{%
% \text{BeginSchroederWick} \text{StartWick} #2 \text{SizeWick} #1 \text{EndSchroederWick}%
% }
% \newcommand{\coC}[1]{%
% \text{BeginSchroederWick} \text{MiddleWick} #1 \text{EndSchroederWick}%
% }
% \newcommand{\enC}[2]{%
% \text{BeginSchroederWick} \text{FinishWick} #2 \text{SizeWick} #1 \text{EndSchroederWick}%
% }


% The file has dual purpose:
% - when placed in the same folder as the tex2html program, it contains only "Html specific commands", and no "Pdf specific commands" . Tex2html will make use of the "Html commands" and generate an html file.
% - when a pdf file is generated, the program automatically concatenates this file with pdf_commands.tex, which contains "Pdf specific commands". The resulting combined file *must* also be called html_commands.tex (so that it would be recognized by main_file.tex). The program then places the resulting file html_commands.tex obtained after concatenation in the same folder as main_file.tex. Then it generates a pdf file by simply running pdflatex in the usual manner (Pdf commands override some of the Html commands to generate a good pdf). Once can then also open the main_file.tex in a tex editor and generate a pdf as usual.
% Html commands will therefore be present in either case, but Pdf commands will only be present in the file located in the same folder as main_file.tex


% Short-hand for labeling things using hyperref (first is for numbering equations, second is for the rest: figures, sections etc).
\newcommand\hyeqref[1]{\hyperref[#1]{\eqref{#1}}}
\newcommand\hyref[1]{\hyperref[#1]{\ref{#1}}}

% ------- Html specific commands: -------
\renewcommand\cite[1]{[#1]}
\newcommand\citet[1]{[#1]}

% Redefine some common latex commands to work with tex2html. 
\renewcommand\indent{IndentHere }
\let\oldbs=\boldsymbol
\renewcommand\boldsymbol[1]{\mathit{\mathbf{#1}}}
%\newcommand\bs[1]{\boldsymbol{#1}}
\let\bs=\boldsymbol
\newcommand\DrawHLine{DrawHLine}
%\renewcommand\[{\begin{equation*}}
%\renewcommand\]{\end{equation*}}

% Usage: just like \caption but for videos. Writes Video: rather than Figure: in the beginning. 
\newcommand\videocaption[1]%
{\renewcommand{\figurename}{Video}%
\caption{#1}
}

% For including videos (similarly to figures).
% Usage: \includevideo{name} . Use same as you would use \includegraphics but do *NOT* write an extension. File must be name.ogg to be viewed with Firefox and name.mp4 to be viewed with Safari. Example:
%\begin{figure}[h!] \label{mov_bear}
%\centering
%\includevideo{bear_movie}
%\caption{Some movie of a bear.}
%\end{figure}
\newcommand\includevideo[1]%
{BeginVideo #1.ogg%
\includegraphics{fig_1.png}%
EndVideo
}

% Toggle environment
% Usage: \begin{htmlToggle}[What to show/hide]{Show/Hide}{label}[background color] Text to be shown \end{htmlToggle}
% Color codes: 6 hexadecimal numbers representing RGB. Eg. red = FF0000, yellow = FFFF00, black = 000000 etc  (see http://html-color-codes.info/)
% Here you can define your own color codes as below
\usepackage{xargs}
\def \White{FFFFFF}
\def \ZerothGray{FDFDFD}
\def \FirstGray{FAFAFA}
\def \SecondGray{F8F8F8}
\def \ThirdGray{F2F2F2}
\def \MyRed{FF0000}
\newenvironmentx{htmlToggle}[4][1=null, 4=\ZerothGray]% ----> SET DEFAULT COLOR HERE!!
{BeginToggle #2 #3 #4 #1 BeginToggle}%
{EndToggle}

% Alert environment
% Usage: \begin{htmlAlert}{Alert contents} Stuff to click on \end{htmlAlert}
% Examples may include math, links, text of several lines etc:
%\begin{htmlAlert}{\href{myhtml.html}{Url} } Click here! \end{htmlAlert} \\
%\begin{htmlAlert}{\href{http://www.google.com}{Google!} } Click here! \end{htmlAlert} \\
%\begin{htmlAlert}{Some text\\ some more boldsymbol $\boldsymbol{x}\boldsymbol{y}$  \\ and more and more and more and more and more and more and more and more and more and more and more and more and more } Click me! \end{htmlAlert} \\
%\begin{htmlAlert}{$x^\mu\arr x\prime^\mu$} Click me too! \end{htmlAlert} \\
\newenvironment{htmlAlert}[1]% 
{ BeginAlert #1 AlertOnClick}%
{EndAlert}

% Surprisingly this had not been defined. So we're defining it here.  
\newenvironment{fulljustify}%
{FullJustify}%
{EndJustify}

% Create multiple columns in the html file.
% Usage: \begin{htmlColumns}{# of columns} Text to be placed in columns \end{htmlColumns}
\newenvironment{htmlColumns}[1]%
{BeginColumns #1}%
{EndColumns  \DrawHLine}

%TODO: These are not added to the code yet. Will follow soon.
% % Wick contraction using "simplewick" package. Redefine simplewick's command "contraction" to work for html.
% \usepackage{simplewick}
% \let\oldContraction\contraction
% \renewcommand{\contraction}[5][5px]{%
%  \text{BeginWick}  #2 \text{StartWick}  #3 \text{MiddleWick}  #4 \text{FinishWick}  #5 \text{SizeWick} \text{#1} \text{EndWick}%
% }
% \newenvironment{simplewick}%
% {\text{BeginSimpleWick}}%
% {\text{EndSimpleWick}}
% 
% % Wick contraction using Dan Schroeder Macros.
% \newcommand{\beC}[2]{%
% \text{BeginSchroederWick} \text{StartWick} #2 \text{SizeWick} #1 \text{EndSchroederWick}%
% }
% \newcommand{\coC}[1]{%
% \text{BeginSchroederWick} \text{MiddleWick} #1 \text{EndSchroederWick}%
% }
% \newcommand{\enC}[2]{%
% \text{BeginSchroederWick} \text{FinishWick} #2 \text{SizeWick} #1 \text{EndSchroederWick}%
% }


% The file has dual purpose:
% - when placed in the same folder as the tex2html program, it contains only "Html specific commands", and no "Pdf specific commands" . Tex2html will make use of the "Html commands" and generate an html file.
% - when a pdf file is generated, the program automatically concatenates this file with pdf_commands.tex, which contains "Pdf specific commands". The resulting combined file *must* also be called html_commands.tex (so that it would be recognized by main_file.tex). The program then places the resulting file html_commands.tex obtained after concatenation in the same folder as main_file.tex. Then it generates a pdf file by simply running pdflatex in the usual manner (Pdf commands override some of the Html commands to generate a good pdf). Once can then also open the main_file.tex in a tex editor and generate a pdf as usual.
% Html commands will therefore be present in either case, but Pdf commands will only be present in the file located in the same folder as main_file.tex


% Short-hand for labeling things using hyperref (first is for numbering equations, second is for the rest: figures, sections etc).
\newcommand\hyeqref[1]{\hyperref[#1]{\eqref{#1}}}
\newcommand\hyref[1]{\hyperref[#1]{\ref{#1}}}

% ------- Html specific commands: -------
\renewcommand\cite[1]{[#1]}
\newcommand\citet[1]{[#1]}

% Redefine some common latex commands to work with tex2html. 
\renewcommand\indent{IndentHere }
\let\oldbs=\boldsymbol
\renewcommand\boldsymbol[1]{\mathit{\mathbf{#1}}}
%\newcommand\bs[1]{\boldsymbol{#1}}
\let\bs=\boldsymbol
\newcommand\DrawHLine{DrawHLine}
%\renewcommand\[{\begin{equation*}}
%\renewcommand\]{\end{equation*}}

% Usage: just like \caption but for videos. Writes Video: rather than Figure: in the beginning. 
\newcommand\videocaption[1]%
{\renewcommand{\figurename}{Video}%
\caption{#1}
}

% For including videos (similarly to figures).
% Usage: \includevideo{name} . Use same as you would use \includegraphics but do *NOT* write an extension. File must be name.ogg to be viewed with Firefox and name.mp4 to be viewed with Safari. Example:
%\begin{figure}[h!] \label{mov_bear}
%\centering
%\includevideo{bear_movie}
%\caption{Some movie of a bear.}
%\end{figure}
\newcommand\includevideo[1]%
{BeginVideo #1.ogg%
\includegraphics{fig_1.png}%
EndVideo
}

% Toggle environment
% Usage: \begin{htmlToggle}[What to show/hide]{Show/Hide}{label}[background color] Text to be shown \end{htmlToggle}
% Color codes: 6 hexadecimal numbers representing RGB. Eg. red = FF0000, yellow = FFFF00, black = 000000 etc  (see http://html-color-codes.info/)
% Here you can define your own color codes as below
\usepackage{xargs}
\def \White{FFFFFF}
\def \ZerothGray{FDFDFD}
\def \FirstGray{FAFAFA}
\def \SecondGray{F8F8F8}
\def \ThirdGray{F2F2F2}
\def \MyRed{FF0000}
\newenvironmentx{htmlToggle}[4][1=null, 4=\ZerothGray]% ----> SET DEFAULT COLOR HERE!!
{BeginToggle #2 #3 #4 #1 BeginToggle}%
{EndToggle}

% Alert environment
% Usage: \begin{htmlAlert}{Alert contents} Stuff to click on \end{htmlAlert}
% Examples may include math, links, text of several lines etc:
%\begin{htmlAlert}{\href{myhtml.html}{Url} } Click here! \end{htmlAlert} \\
%\begin{htmlAlert}{\href{http://www.google.com}{Google!} } Click here! \end{htmlAlert} \\
%\begin{htmlAlert}{Some text\\ some more boldsymbol $\boldsymbol{x}\boldsymbol{y}$  \\ and more and more and more and more and more and more and more and more and more and more and more and more and more } Click me! \end{htmlAlert} \\
%\begin{htmlAlert}{$x^\mu\arr x\prime^\mu$} Click me too! \end{htmlAlert} \\
\newenvironment{htmlAlert}[1]% 
{ BeginAlert #1 AlertOnClick}%
{EndAlert}

% Surprisingly this had not been defined. So we're defining it here.  
\newenvironment{fulljustify}%
{FullJustify}%
{EndJustify}

% Create multiple columns in the html file.
% Usage: \begin{htmlColumns}{# of columns} Text to be placed in columns \end{htmlColumns}
\newenvironment{htmlColumns}[1]%
{BeginColumns #1}%
{EndColumns  \DrawHLine}

%TODO: These are not added to the code yet. Will follow soon.
% % Wick contraction using "simplewick" package. Redefine simplewick's command "contraction" to work for html.
% \usepackage{simplewick}
% \let\oldContraction\contraction
% \renewcommand{\contraction}[5][5px]{%
%  \text{BeginWick}  #2 \text{StartWick}  #3 \text{MiddleWick}  #4 \text{FinishWick}  #5 \text{SizeWick} \text{#1} \text{EndWick}%
% }
% \newenvironment{simplewick}%
% {\text{BeginSimpleWick}}%
% {\text{EndSimpleWick}}
% 
% % Wick contraction using Dan Schroeder Macros.
% \newcommand{\beC}[2]{%
% \text{BeginSchroederWick} \text{StartWick} #2 \text{SizeWick} #1 \text{EndSchroederWick}%
% }
% \newcommand{\coC}[1]{%
% \text{BeginSchroederWick} \text{MiddleWick} #1 \text{EndSchroederWick}%
% }
% \newcommand{\enC}[2]{%
% \text{BeginSchroederWick} \text{FinishWick} #2 \text{SizeWick} #1 \text{EndSchroederWick}%
% }

